\chapter{Compiler Structure}
\label{chapter:compiler structure}

As already hinted by the abstract, a compiler consists of a front-end and a back-end, but also a third part that is the middle-end. These three parts sit on top of each other with the front-end on top and the back-end at the bottom and pass down the programs code as it is translated and optimized or compiled. But communication between the parts does not go only one way and changes that are made in the back-end affect the front-end as well!
The first part of the compilation process is the translation of code which is written in some programming language into a so called Immediate Representation (IR) that looks the same for every front-end language and usually is never seen by the user. Any supported programming language (C, C++, Java…) is implemented in its own front-end that defines how the language is translated into IR. After that the IR is send to the middle-end, which generally optimizes the IR and then passes the code to the back-end. The back-end first executes further optimizations that are target-specific followed by allocating registers and handling relative memory. Finally the code is translated into the assembly language that is supported by the target.
After the code is compiled and emmited as an object file it is also linked, which means combining different objectfiles and assigning absolute memory addresses to them. At last the binary file emmited by the linker is loaded into the memory of the processor to be executed.
